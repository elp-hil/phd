\chapter{Materials and Methods}\label{chap:MaterialsAndMethods}
	\section{Low field NMR}
		To achieve NMR spectra at fields where SABRE is feasible \todo{Ref sabre}, a low field NMR
		spectrometer was built \todo{ref niels}. Its main field is genrated by a resisitve solenoid
		coil. Inside that coil, there is a saddle coil
		generating a $B_1$ field perpendicular to $B_0$. Perpendicular to and inside both, a third
		coil, also a solenoid, is used to detect the signal generated by the spins.
		\subsection{Static Magnetic Field}
			The $B_0$ coil is wound around an acrylic tube in two full layers. In addition, at the
			tube's ends compensation windings are installed to homogenize the field inside the coil.
			The length of these windings was optimized in matlab simulations \todo{figure of whole setup}
		\subsection{Radiofrequency Excitation}
			To irradiate samples with radiofrequency pulses, a saddle coil \todo{dimensions} wsa
			used. It was operated untuned and unmatched as a broadband resonator. The pulse
			generation was performed using a National Instruments data acquisition crate (NI
			\todo{which?}). 
		\subsection{Software control}
			All software control was realized using Matlab in combination with the DAQmx libraries. 
		\subsection{Data Readout}
			All readout was done using a NI \todo{name}
		\subsection{Shim System}
			For homogenization of the field, a shim system was built according to Biot Savart
			simulations. It features linear shim coils for all three spatial dimensions mounted to a
			.\todo{cm} acrylic tube. The x and y shims are made of four saddle coils respectively that
			were plainly manufactured individually and bent to fit the tube. The z shims, which are
			basically a pair of maxwell coils, were added on top of these saddle coils. All shims
			are driven by a \todo{H\&U} programmable power supply providing up to
			$\SI{10}{\ampere}$ of current. The power supply is connected via a virtual serial port inside
			a USB connection. That way, the three shim channels can be controlled from inside the
			Matlab hypercontrol program \ref{subsec:hypercontrol}
	\section{Magritek Low Field MRI}
		To acquire images at low fields, a Magritek Terranova \todo{ref } was used. It features
		similar hardware as the low field spectrometer, but uses its $B_0$ coil only for
		prepolarization while signal is read out at earth magnetic field.
		%\subsection{
		\todo{subsections}
	\section{Bruker Low Field MRI}
		\subsection{Gradient Coil Setup}
		\todo{subsections}
	\section{High field MRI}
		The most well known application of NMR is the high field MRI of human antomy with its
		widespread use in clinics around the world. Not as common, but equally important are
		preclinical scanners for research purposes. These preclinical scanner, primarily built for
		animal experiments, were used in most of the high field experiments shown in this work.
		\subsection{MRI Hardware}
			
		\subsection{Paravision Software}
		\subsection{Custom High Field Coils}
			Most commercially available coils are for proton imaging and spectroscopy. Coils for
			other nuclei are obtainable, but usually expensive and not necessarily tailored to the
			specific purpose in mind. Therefore, we built single and dual tune coils for different
			nuclei and different fields.
			\subsubsection{$^{15}N$ coil}
				A solenoid of thick, stable copper wire was wound to fit the experimental setup of
				the shuttling system described in \ref{sec:shuttlingSystem}. The solenoid was
				attached to a circuit board via clamped and soldered connections. On the board, a
				high voltage tune capacitor as well as two symmetric matching capacitors were
				installed. Coaxial cable was used to make the connection to the scanner and the
				whole setup was mounted to a teflon holder for precise positioning.
				The tune capacitor was chosen so that it can be tuned to both a $\SI{7}{\tesla}$ and
				a $\SI{9.4}{\tesla}$ field at $\SI{300}{\MHz}$ and $\SI{400}{\MHz}$ respectively.
				\todo{image coil}
				
	\section{Shuttling system}\label{sec:shuttlingSystem}
		\subsection{Magnetic Shielding}
		\subsection{Low Field Reactor}
		\subsection{High Field Probe}
		\subsection{Fluid Handling System} 
	\section{Fluxgate Field Probe}
		\subsection{Arduino Shield}
		\subsection{}
\todo{fill with stories of my life}
%\input{}
