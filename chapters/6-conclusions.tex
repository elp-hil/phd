% !TEX root = ../thesis_main.tex
\chapter{Discussion and Conclusion}
    \label{chap:conclusion}
    \section{Parahydrogen generator and generation}
        The generator previously built in the group worked well and performed at least as good as commercially available devices. The parahydrogen enrichment was sufficient for the measurements performed and turned out to be reliably high whenever randomly sampled.
        Filling bottles of parahydrogen on the roof to then use in the lab is a well established practice. Using the new 15N setup, where high pressures and flows were used, this posed a problem though as bottle volumes often were too small to provide parahydrogen for long enough and showed pressure drops in longer measurements. A steady flow inline pH2 generator would be preferable for this appliance. As it was ruled out iat first because of security reason, this should be reconsidered in future. As a intermediate solution to the problem, larger pH2 bottles were ordered (\SI{5}{\l} and \SI{10}{\l}) that provide the right pressures for longer.
    \section{Low field spectrometer}
        The previously built low field spectrometer was used in many of the different measurements described in this work. More coils were built - on all fronts - i.e. field generation, transmit and receive coils. A discussion on what was imporved, what was kept is coming up next.
    \subsection{$B_0$ field generation}
    Multiple coil designs of the static field generating coils have been considered, simulated, built and tested in the course of this work. The previously used solenoid design with different lengths of compensation windings (section \ref{sec:results:sim:B0}) showed rather broad lines in both simulations and measurements. Main reason are the discrete number of compensation windings that do not allow for fine enough tuning of the coil. This could be improved by putting the compensation windings on sliders that can move in z-Direction. additionally in the design used, mechanical errors in the build worsened the problematic. Due to the high number of windings guided only by the previously wound wire, shifts building along the coil's z axis are inevitable. In addition, layering the wire leads to slippage into the gouge created by the previous layer. An additional test using rectangular wire instead of round for both better filling factors and more precise winding was too difficult in terms of manufacture. The problem may be solved by using a thin, but stiff layer of material to separate the wire layers from each other. All of the mentioed solutions are inconvenient though when considering the dimensions of the coil and different approaches such as the dual helmholtz coil mentioned later in this section had to be considered.
        The heating of the solenoid coil led to an additional problem: Field shifts, probably due to the thermal expansion of the copper heating by $\Delta T = \SI{40}{\kelvin}$ for the \SI{35}{\centi\meter} long coil leads to an expansion of 
        \begin{equation}
            \Delta x \approx \alpha \mathrm{L}\Delta\mathrm{T}= \SI{16.5e-6}{\per\kelvin}\cdot \SI{40}{\kelvin} \cdot \SI{0.35}{\meter} = \SI{0.2}{\milli\meter}
        \end{equation}
        This elongation induces a clearly visible frequency shift that can be problematic for measurement reproducibility (compare also to simulations of sub-milimeter positioning errors, figure \ref{fig:results:fieldSpread}).
        To avoid field shifts during measurements, the setup should therefore be in thermal equilibrium. This is especially relevant when switching fields quickly during measurements using the programmable power supply after which this equilibrium is usually shifted to a different temperature. After every switchng, a drift towards the new equilibrium can be observed via the frequency.
        The dual helmholtz array design deals with many of the above mentioned problems. The single coils are wound onto a milled holder making moving in z direction possible and convenient. Submillimeter precision was strived for and experimentally verified through the narrow lines corresponding to the simulation results. Due to the shorter extension of the coils in z-direction in combination with the precisely milled holders, the errors introduced in the winding process are smaller than for the solenoid. Using PVC foil to separate the axial layers kept them from slipping into the gaps of the previously wound layers adding to both radial and axial precision. The rather narrow, lasercut PVC layers were easily installed compared to the large area foil necessary for a more precise solenoidal design.
        In addition, the more open design allows for axial access to the sample which can make experiments a lot more convenient. The disadvantage of the design are the higher currents due to the lower overall winding count combined with larger distances to the measured objekt that make need for higher current power supplies. In this case, i.e. the \SI{5}{\milli\tesla} field generation, a easily commercially available, four channel, 32V/10A power supply was sufficient though (section \ref{}).
    \subsection{$\mathrm{B}_1$ coils}
    The $B_1$ coils show a broadband behaviour that was intended and expected. 180 degree flip angles were well possible if the coil voltage was not too high, i.e. if the pulses were long enough. Exceeding the voltage limit causes arcing especially inside the capacitors and thus cuts the effective coil voltage at the sine peaks. Additional effects such as ionization may lead to the behavior shown in chapter \todo{ref}, i.e. a more a less constant flip angle, but with a large error.
    The matching of the $\mathrm{B_1}$ coil depended on its geometric orientation towards the receive coil due to inductive coupling of the two.
    Experiments with printed circiuts on copper foil for etching were unsuccessful due to printer problems, but may greatly improve the coil's geometric accuracy compared to the rather complicated winding process used up to date. The design for these circuits can be found in \todo{appendix, ref}
    \subsection{Receive coils}
    The resonances measured with a dedicated network analyzer showed peaks very different from the resonances measured with the NI card used in the exeriments. As the more relevant resonance frequency is the actual frequency during measurements, the NI-card's result was used to setup the coils. It has to be considered though that the network analyzer provides a more realistic result of the coil's intrinsic resonance frequency. For future builds, network analyzer results and calculations can be considered for a first estimation, but fine tuning has to be done with the measurement card itself. Generally, it would be beneficiary to use the higher Q resonance shown in figures \ref{fig:results:networkAnalysis} and \ref{fig:results:niNetworkAnalysis}, but the behaviour described above lead to the use of the lower Q resonance during the measurements performed in this work.
    While single channel coils are well suited for low field experiments, dual channel coils often show lower performance and sensitivity on at least one of the two channels and a two channel setup is often preferrable to a dual channel one (see methods, \ref{}). The determined Q-factors were well within the range that delivers acceptable SNR and does not overcharge the amplifier. Generally, the audio amplifier used was at its power limit and prone to overpower-shutdowns - for future setups, a dedicated amplifier delivering the power necessary more easily has to be considered.
    \subsection{Shims}
    The linear shims built for the low field unit work as intended and reduce linewidths significantly. For future builds, higher order shims should be considered as the field distribution of the solenoid coils - considering both simulations and linewidths - seemed to have large quadratic fractions. Thus, linear shims did improve liewidths slightly but were not able to get beyond linewidths of \SI{50}{\hertz}. Considering the low fields, this is rather large with ~200 ppm. This point is supported by the fact that the two more parabolically shaped fields of the Helmholtz pairs provide narrower lines to begin with, without additional shims (although one could consider one helmholtz pair as the field generating one and one as a quadratic shim). The shim tool implemented into the existing matlab data recording tool is working well and finds the same settigns for minimal linewidths from different starting points and trough different iterations.
    \section{In-situ Sabre in water}
    The method described by \ref{} was successfully implemented on our low field spectrometer in a continuous fashion. By doing so, buildup times could be measured and
    \subsection{Buildup times}
        Knowledge of the buildup times is key for maximizing the signal output in consecutive measurements, similar to the Ernst angle.
    \subsection{Sabre in cell culture solution}
    \subsection{Sabre in blood}
        The fact that signal is depletet upon contact with blood very quickly and very strongly, even with the low amounts of blood used in this work makes the vision of in vivo measurements vanish out of view. 
    \section{Imaging at earth magnetic field}
        While the signal of the hyperpolarized solution and thus its SNR was high compared to the water sample, it has to be considered that the prepolarization magnetic field was only \SI{5}{\milli\tesla} and thus about three orders of magnitude below usual commercially available MRI. At higher magnetic fields for readout, hyperpolarized signal would remain fairly constant (neglecting changed relaxation effects). Thermal signal of the pure water wuould increase linearly with the field though and thus, a high field image is to be preferred over the low field hyperpolarized signal considering singnal only. If, though, a metabolic process could be monitored with the hyperpolarizzed nicotinamide, the setup would allow for doing this with the low concentrations provided and fairly low signal background. Furthermore, the signal increase can be used to build cheap low field imagers in specific fields such as transportable machines or low cost machines e.g. for developing countries. Here, the rather complicated procedure has to be considered though and a system that is more automated would be necessary.
    \section{Sabre Shuttling System}
        The system was designed to perform measurements in a well reproducable setting. As all shuttling and even scanner control are automated, volume shifts and timings are very well reproduced within consecutive scans. A measurement of those timings was not performed as the error is surely small compared to the error introduced by fluid loss, fluid flow and bubbles in solution.

        \subsection{Temperature control}
        \label{cd:sabreShuttling:tempControl}
            Temperature is the one factor that is both not controlled and also very difficult to contorl in this setting. Wall thickness has to be in a range that withstands the pressures used in the setup and thus makes all indirect heating through the walls difficult, especially since the fluid voulme is small compared to the PSU volume surrounding it. Temperature measurement would be easily possible through optical, contactless sensors, but temperature control is not solved. Optimal temperatures for the IR-IMes/Methanol/15N-pyridine mixture were shown to be at around \SI{14}{\celsius}. As these temperatures are below laboratory AC temperatures of \SI{18}{\celsius} and the parahydrogen flow decreases the sample's temperature, the optimal temperature was reached (experiments show temperature decreases towards \SI{0}{\celsius} under constant flow), but may well be undercut. As temperature is not controlled or regulated in the current setup, a sleeve for a cooling or heating liquid may be considered in future designs. \todo{fig temp sleeve}.
        \subsection{Automated handvalve}
        The low volume handvalve worked reliably throughout the measurements. As the adaptor connecting servo and valve was forged by hand, collinearity of the axes was not given and led to a slight wobble of the setup during the moving of the valve while straining the fixation screws. Here, a future design should be either 3D-printed or milled to make for a better axial alignment to ensure long term stability.
        \subsection{Shuttling speeds}
        The duration of the shuttling procedure was short enough to keep hyperpolarization at levels sufficient for parameter analysis and polarizations in the range previously reported. For future setups, an additional field that can be switched on before shuttling should be considered as T1 relaxation at \SI{}{\nano\tesla} fields is a lot faster (~ \SI{1}{second}) than at higher fields. Fields of \SI{1}{\micro\tesla} suffice to increase relaxation times (pyridine in methanol) to \todo{times} \SI{10}{\second}. The field could be generated by a wire wound around reactor and transfer capillary carrying a low current. At the time the capillary leaves the mu metal shielding, the stray field of the scanner is large enough to protect the solution from quick relaxation.
        \subsection{Shuttling reproducibility}
            As shown in section \ref{results:15N:shuttlingReproducibility}, the relative error on the shuttling process is low. It has to be considered though that a completely dried system will reduce the amount of substance arriving on the high field side by a non neglectable amount by humidification of the surfaces. That is why, depending on the previous state of the system, a drop in signal could be observed during the first or first two shuttling procedures. After that, the only source of fluid loss is the evaporation of liquid during bubbling - leaks of fluid have not been observed unless a seal failed. High flows generate high polarization  and signal yield up to a certain point. Above that point, probably due to the formation of larger bubbles and streaming channels that are visible outside the shield, the signal does not rise further. At the same time, large losses in fluid volume over short periods of time occur if the flow is too high. Therefore, for parameter optimization, usually  a lower flow rate was chosen to make consecutive scans more comparable in terms of volume and thus signal. The signal scaling with flow should be independent of the other parameters and can therefore be adapted accordingly.
        \subsection{Pressure dependence}
            Polarization shows a linear dependence on pressure in the measured range. This indicates that higher pressures would still increase the signal as a plateau is to be expected where a higher pH2 concentration in solution does not increase reaction rates at the catalyst any more because its concentration is already large compared to the other reactants. Increasing pressures further in future setups is certainly possible considering the still rather thin walls of the PSU reactors. One limitation that occurs are certainly the screwed ferrule connections of the capillary and PTFE tubing which, under high pressures can slip uot of its fit. It can be replaced by more sophisticated flanged ferrule connections which require special tools that are commercially available though. Larger inner diameter tubing also starts to reach its pressure limitations above \SI{50}{\bar}, but can easily be replaced by tubings with smaller inner diameter. Reduced flow through these tubings will, on the one hand, be compensated by the higher pressure differences themselves, on the other hand, the only fluid pathway is already covered by a small diameter capillary resisting pressures up to \SI{200}{\bar} according to its data sheet.
        \subsection{Field dependence}
            The dependence on the magnetic field shows a pattern of two peaks around a minumum with signal going towards zero in the outskirts, i.e. towards high fields. This makes sense as simulations suggest fields around \SI{300}{\nano\tesla} delivering the energy splitting range in which level anti crossings are fully developed. The overall field depends on the residual magnetization of the mu metal shielding and is measured indepently. That makes direct comparisons difficult as the shield needs to be opened to change from measurement head (fluxgate) to the polarization system. During that exchange magnetization of the shield can change by mecahnical stress such as hits to the shields or lids or by externel magentic fields reaching into the inner shields and leaving residue magnetization even after opening.The asymmetry of the distribution is due to residual magnetic field of the shield. It generates both a shift of the 'mirror axis' and an overall deformation of the distribution depending on the field homogeniety.

        \subsection{Concentration dependence}
            The concentration plays an important role in the Sabre complex formation, the formation of different intermediates and also the disassembly of the complexes. Generally, a concentration that is large compared to the catalyst concentration will lead to a reduction in polarization as the pool of substrate molecules in solution is large compared to those in bound form. This can be seen from the data, as a reduction of the relative concentration of susbtrate leads to a signal increase. Catalyst concentration remains constant in the example given. This means that, although the overall amount of substance of the substrate does not change, more molecules are hyperpolarized at the same time and thus a larger magnetization is generated.
        \subsection{Polarization and magnetization}
            Polarization reached levels in the previosly reported ranges while magnetization was high considering the high concentrations of \SI{50}{\milli\molar}. In many publications researchers seem to try and maximize polarization - in this work, magnetization, i.e. absolute signal intensity was considered more important as for any future application, this is the relevant parameter for maximizing the signal. Considering the concentration used here, the polarization surpasses previously reported values and makes the setup suitable for batch polarization of any future use in imaging experiments.
