% !TEX root = ../thesis_main.tex
\chapter{Discussion and Conclusion}\label{chap:conclusion}
    \section{$B_0$ field generation}
        Multiple coil designs of the static field generating coils have been considered, simulated, built and tested in the course of this work. The previously used solenoid design with different lengths of compensation windings (section \ref{sec:results:sim:B0}) showed rather broad lines in both simulations and measurements. Main reason are the discrete number of compensation windings that do not allow for fine enough tuning of the coil. This could be improved by putting the compensation windings on sliders that can move in z-Direction. additionally in the design used, mechanical errors in the build worsened the problematic. Due to the high number of windings guided only by the previously wound wire, shifts building along the coil's z axis are inevitable. In addition, layering the wire leads to slippage into the gouge created by the previous layer. This problem may be solved by using a thin, but stiff layer of material to separate the wire layers from each other. All of the mentioed solutions are inconvenient when considering the dimensions of the coil. 
        The heating of the solenoid coil led to an additional problem: Field shifts, probably due to the thermal expansion of the copper heating by $\Delta T = \SI{40}{\kelvin}$ for the \SI{35}{\centi\meter} long coil leads to an expansion of 
        \begin{equation}
            \Delta x \approx \alpha \mathrm{L}\Delta\mathrm{T}= \SI{16.5e-6}{\per\kelvin}\cdot \SI{40}{\kelvin} \cdot \SI{0.35}{\meter} = \SI{0.2}{\milli\meter}
        \end{equation}
        This change leads to a clearly visible frequency shift that can be problematic for measurement reproducibility (compare also to simulations of sub-milimeter positioning errors, figure \ref{fig:results:fieldSpread})
        To avoid field shifts during measurements, the setup should therefore be in thermal equilibrium. This is especially relevant when switching fields during measurements using the programmable power supply.

        The dual helmholtz array design deals with the above mentioned problems. The single coils are wound onto a milled holder making moving in z direction possible and convenient. due to the shorter extension of the coils in z direction, the errors introduced in the winding process are smaller and using PVC foil to separate the axial layers kept the layers neatly wound.
        In addition, the design allows for axial access to the sample which can make experiments a lot more convenient. The disadvantage of the design are the higher currents due to the lower winding count that make need for higher current power supplies.
    \section{$B_1$ coils}
        The $B_1$ coils show expected behaviour. While single channel coils are well suited for low field experiments, dual channel coils often show low performance and sensitivity on at least one of the two channels and a two coil setup can often be more reasonable. The determined Q-factors were in the range previously mentioned in literature. 
    \section{Shims}
        Shims do work as intended. For future builds, higher order shims should be considered as the field distribution of the solenoid coils was - considering both simulations and linewidths - seemed more quadratic. Thus, linear shims did improve liewidths slightly but were not able to get beyond linewidths of \SI{50}{\hertz}. Considering the low fields, this is rather large with ~200 ppm.
    \section{Imaging at earth magnetic field}
        While the signal of hyperpolarized solution was high compared to the water sample, it has to be considered that the prepolarization magnetic field was only \SI{5}{\milli\tesla} and thus about three orders of magnitude below usual commercially available MRI. At higher magnetic fields for readout, hyperpolarized signal would remain fairly constand (neglecting changed relaxation effects). Thermal signal of the pure water wuould increase linearly with the field though and thus, a high field imager is to be preferred over a low field hyperpolarized signal considering singnal only. If, though, a metabolic process could be monitored with the hyperpolarizzed nicotinamide, the setup would allow for doing this with the low concentrations provided and fairly low signal background. Furthermore, the signal increase can be used to build cheap low field imagers in specific fields such as transportable machines or low cost machines e.g. for developing countries. Here, the rather comlicated procedure has to be considered though and a system that is more automated would be necessary.
    \section{Sabre Shuttling System}
        The system was designed to perform measurements in a well reproducable setting. As all shuttling and even scanner control are automated, volumes and timings are very well reproduced within consecutive scans.
        \subsection{Temperature control}
        \label{cd:sabreShuttling:tempControl}
            Temperature is the one factor that is both not controlled and also very difficult to contorl in this setting. Wall thickness has to be in a range that withstands the pressures used in the setup and thus makes all indirect heating through the walls difficult, especially since the fluid voulme is small compared to the PSU volume surrounding it. Temperature measurement would be easily possible through optical, contactless sensors, but temperature control is not solved. Optimal temperatures for the IR-IMes/Methanol/15N-pyridine mixture were shown to be at around \SI{14}{\celsius}. As these temperatures are below laboratory AC temperatures of \SI{18}{\celsius} and the parahydrogen flow decreases the sample's temperature, the optimal temperature was reached (experiments show temperature decreases towards \SI{0}{\celsius} under constant flow), but may well be undercut. As temperature is not controlled or regulated in the current setup, a sleeve for a cooling or heating liquid may be considered in future designs. \todo{fig temp sleeve}.
        \subsection{Shuttling speeds}
            The duration of the shuttling procedure was short enough to keep hyperpolarization at levels sufficient for analysislevels. For future setups, a additional field that can be switched on before shuttling starts should be considered as T1 relaxation at \SI{}{\nano\tesla} fields is a lot faster (~ \SI{1}{second}). Fields of \SI{1}{\micro\tesla} suffice to increase relaxation times (pyridine in methanol) to \todo{times} \SI{10}{\second}. 
        \subsection{Shuttling reproducibility}
            As shown in section \ref{results:15N:shuttlingReproducibility}, the relative error on the shuttling process is low. It has to be considered thoug that a completely dry system will reduce the amount of substance arriving on the high field side by a non neglectable amount by humidification of the surfaces. That is why, depending on the previous state of the system, a drop in signal could be observed during the first or first two shuttling procedures. After that, the only source of loss is the evaporation of liquid during bubbling. High flows generate high polarization and signal yield up to a certian point, but also cause large losses in fluid volume over short periods of time. Therefore, for parameter optimization, usually  a lower flow rate was chosen to make consecutive scans more comparable. The scaling with flow should be independent of the other parameters and can therefore be adapted accordingly.
        \subsection{Pressure dependence}
            Polarization shows a linear dependence on pressure in the measured range. This indicates that higher pressures would still increase the signal as a plateau is to be expected where a higher pH2 concentration in solution does not increase reaction rates at the catalyst any more because it already is steadily available. Increasing pressures further in future setups is certainly possible considering the still rather thin walls of the PSU reactors. One limitation that occurs are certainly the screwed ferrule connections of the capillary and PTFE tubing which, under high pressures can slip uot of its fit. It can be replaced by more sophisticated flanged ferrule connections which require special tools that are commercially available though. Larger inner diameter tubing also starts to reach its pressure limitations above \SI{50}{\bar}, but can easily be replaced by tubings with smaller inner diameter. Reduced flow through these tubings will, on the one hand, be compensated by the higher pressure differences themselves, on the other hand, the only fluid pathway is already covered by a small diameter capillary resisting pressures up to \SI{200}{\bar} according to its data sheet.
        \subsection{Field dependence}
            The dependence on the magnetic field shows a pattern of two peaks around a minumum with signal going towards 0 in the outskirts, i.e. towards high fields. This makes sense as simulations suggest fields around \SI{300}{\nano\tesla} delivering the energy splitting range in which level anti crossings are fully developed. The overall field depends on the residual magnetization of the mu metal shielding and is measured indepently. That makes direct comparisons difficult as the shield needs to be opened to change from measurement head (fluxgate) to the polarization system. During that exchange magnetization of the shield can change by mecahnical stress such as hits to the shields or lids or by externel magentic fields reaching into the inner shields and leaving residue magnetization even after opening.The asymmetry of the distribution is due to residual magnetic field of the shield. It generates both a shift of the 'mirror axis' and an overall deformation of the distribution depending on the field homogeniety.

        \subsection{Concentration dependence}
            The concentration plays an important role in the Sabre complex formation, the formation of different intermediates and also the disassembly of the complexes. Generally, a concentration that is large compared to the catalyst concentration will lead to a reduction in polarization as the pool of substrate molecules in solution is large compared to those in bound form. This can be seen from the data, as a reduction of the relative concentration of susbtrate leads to a signal increase. Catalyst concentration remains constant in the example given. This means that, although the overall amount of substance of the substrate does not change, more molecules are hyperpolarized at the same time and thus a larger magnetization is generated.
        \subsection{Polarization and magnetization}
            Polarization reached levels in the previosly reported ranges while magnetization was high considering the high concentrations of \SI{50}{\milli\molar}. Many publications seem to try and maximize polarization - in this work, magnetization, i.e. absolute signal intensity was considered more important as for any future application, this is the more relevant parameter.
