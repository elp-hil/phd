% !TEX root = ../thesis_main.tex
\chapter{Conclusion}\label{chap:conclusion}
	\section{$B_0$ field generation}
		Multiple coil designs of the static field generating coils have been considered, simulated, built and tested in the course of this work. The previously used solenoid design with different lengths of compensation windings \ref{simulation:B0} showed rather broad lines in both simulations and measurements. Main reason are the discrete number of compensation windings that do not allow for fine enough tuning of the coil. This could be improved by putting the compensation windings on sliders that can move in z-Direction. additionally in the design used, mechanical errors in the build worsened the problematic. Due to the high number of windings guided only by the previously wound wire, shifts building along the coil's z axis are inevitable. In addition, layering the wire leads to slippage into the gouge created by the previous layer. This problem may be solved by using a thin, but stiff layer of material to separate the wire layers from each other. All of the mentioed solutions are inconvenient when considering the dimensions of the coil.
		
		The dual helmholtz arary design deals with the above mentioned problems. The single coils are wound onto a milled holder making moving in z direction possible and convenient. due to the shorter extension of the coils in z direction, the errors introduced in the winding process are smaller and using PVC foil to separate the axial layers kept the layers neatly wound.
		In addition, the design allows for axial access to the sample which can make experiments a lot more convenient. The disadvantage of the design are the higher currents due to the lower winding count that make need for higher current power supplies.
	\section{$B_1$ coils}
		
