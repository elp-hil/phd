\chapter{Theory}\label{chap:theory}
	\section{Nuclei and Spin}
	\section{NMR}
	Nuclear Magnetic Resonance (NMR) is a technique that emerged in 1946 with the discovery
	of the absortion properties of nuclei irradiated with electromagnetic waves resonant to their
	larmor frequencies.\ref{ReosnanceAbsorption} The subsequent observation of signal from those
	previously excited nuclei, known as free induction decay (FID) paved the road to the now well
	established method which in the beginning was primarily used in chemistry for structural
	analysis of Molecules and chemical kinetics.
		\subsection{Flip angle}
			An external, on-resonant magnetic field will cause the magnetization of an ensemble of
			spins to flip by an angle $\alpha$ known as the flip angle (FA). An FA of $n\cdot 180
			\deg + 90 \deg$ will rotate the magnetization to the transverse plane perpendicular to the magnetic field
			resulting in an FID. Pulses of $n\cdot 180 \deg$ will keep the magnetization aligned
			with the magnetic field or invert it, therefore not resulting in a FID. All other FAs
			will produce a linear combination of the two cases.
		\subsection{Relaxation}
			After being deflected from its equilibrium, z-magnetization tends to relax towards its
			thermal equilibrium. That process is called T1 relaxation or spin-lattice relaxation and
			is an exponential decay. In addition, the transverse magnetization decays - usually much
			faster - because of different magnetic fields experienced by the individual spins
			leading to a dephasing of the signal.
			\begin{figure}{h}
				\todo{T1/T2 figure}
			\end{figure}
		\subsection{A simple NMR experiment}
			The most basic experiment in NMR is the reaction of a spin ensemble to an 
		\subsection{Field Gradients}
		\subsection{2D-NMR}
	\section{MRI}
		\subsection{MRI in Medicine}
		\subsection{Frequency encoding}
		\subsection{Phase encoding}
	\section{Hyperpolarization}
		The main limitation of NMR and MRI is the low thermal polarization.
		For each energy state $E_i$ the Boltzmann distribution dictates
		\begin{equation}
			N_i = \exp{\frac{E_i}{k_B T}}
		\end{equation}
		For energy differences between two states that are small compared to
		the thermal Energy, the polarization can be expressed as follows:
		\begin{equation}
			P = {N_+-N_-}{N} = \tanh\left(\frac{\hbar \gamma B}{2 k T }\right)
		\end{equation}
		\subsection{DNP}
		\subsection{Hyperpolarization of Noble Gases}
		\subsection{Brute Force Hyperpolarization}
		\subsection{Parahydrogen induced Hyperpolarization}
