\chapter{Theory}\label{chap:theory}
	\section{Nuclei and Spins}
	The postulation of an electron spin in 1926 \ref{} and later in 1928 of a proton spin \ref{} opened
	up new areas of research.\todo{more history}
	Individual spins can be described by their angular momentum operators $\hat{I}$,
	$\hat{I}_y$ and $\hat{I}_z$ that return the eigenvalue of the state in its respective
	direction.
	\begin{equation}
		\hat I_z \ket{I, S} = S \ket{I,S}
	\end{equation}
	Considering a spin-1/2 particle in a magnetic field along the z-axis (e.g. the proton as a prominent particle in NMR), these
	eigenstates are
	\begin{equation}
	\ket\alpha = \begin{pmatrix}1\\0\end{pmatrix} \hspace{2 cm} \ket\beta =
	\begin{pmatrix}0\\1\end{pmatrix}
	\end{equation}
	with the eigenvalues of $\pm 1/2$.
	Generally, every spin-1/2 particle can be in any superposition of those two eigenstates:
	\begin{equation}
		\ket\psi = c_\alpha \ket\alpha + c_\beta \ket \beta = \begin{pmatrix} c_\alpha \\
		c_\beta\end{pmatrix}
	\end{equation}
	These superposition states evolve under external conditions until a eigenstate of the particle
	is reached.
	Generally, a sample does not consist of one, but of many nuclei and thus many spins need to be
	described to describe the system. To do so, usually the density matrix formalism is introduced.
	It onsiders an ensembe of spins that do not interact with each other. In this case, the
	expectation value of an operator for a single spin $\braket{\hat Q}$ is given by
	\begin{equation}
	\bra\psi\hat Q\ket\psi = \left( c_\alpha^*, c_\beta^*\right)
	\begin{pmatrix}
		Q_{\alpha\alpha} Q_{\alpha\beta}\\
		Q_{\beta\alpha} Q_{\beta\beta}
	\end{pmatrix}
	\begin{pmatrix}
		c_\alpha\\
		c_\beta
	\end{pmatrix}
	\end{equation}
	It can be shown that this is equal to the trace of the density operator
	$\ket\psi\bra\psi$multiplied with said operator
	\begin{equation}
		\braket{\hat Q} = \Tr \left\{\ket\psi\bra\psi\hat Q\right\}
	\end{equation}
	For multiple spins, it follows that the observable becomes the sum of their individual
	observables:
	\begin{equation}
		\braket{\hat Q_{all}}= \bra{\psi_1}\hat Q\ket{\psi_1} + \bra{\psi_2}\hat Q\ket{\psi_2} + \hdots =
		\Tr{\left\{\left(\ket{\psi_1}\bra{\psi_1} + \bra{\psi_2}\ket{\psi_2}+ \hdots \right) \hat Q\right\}}
	\end{equation}
	This means that the system of spins can be described by the sum of the density operators called
	the density matrix
	\begin{equation}
		\hat\rho = \overline{\ket\psi\bra\psi}.
	\end{equation}
	For a spin-1/2 ensemble, the density matrix in thermal equilibrium is
	\begin{equation}
		\hat \rho = \begin{pmatrix} \frac{1}{2}+\frac{1}{4}\mathbb{B}& 0\\ 0&
		\frac{1}{2}-\frac{1}{4}\mathbb{B}\end{pmatrix} = \frac {1}{2} \hat1 + \frac{1}{2} \mathbb{B}
		\hat I_z
	\end{equation}
	with $\mathbb{B} = \frac{\hbar\gamma B_0}{k_b T}$. That means that in thermal equilibrium, the
	non-diagonal elements are zero, i.e. the socalled coherences (off-diagonal elements) are all
	equally populated while there is a slight overpopulation of one of the two states. Through deflection from that
	equilibrium, coherences are populated as described in the following section.
		\subsection{Radiofrequency pulses}
		Using the density matrix formalism, radiofrequency pulses described by rotational operators can be
		applied to the whole ensemble. A pulse $\hat R_\phi(\beta)$ of phase $\phi$ (corresponding
		to the axis around which magnetization is rotated) and angle $\beta$ where
		$\beta=\omega_{nut} \cdot \tau_p$ exerting on a state $\ket\psi$ is described by 
		\begin{equation}
			\ket{\psi_\tau}= \hat R_\phi(\beta)\ket\psi
		\end{equation}
		Calculating the density matrix now leads to
		\begin{equation}
			\hat {\rho_\tau} = \overline{\ket{\psi_\tau}\bra{\psi_\tau}} = \overline{\hat
				R_{\phi}(\beta)\ket\psi\bra\psi \hat R_\phi(-\beta)}
		\end{equation}
		where the overbar describes the averaging over all spins in the ensemble.
		Finally, as we are considering the same flip angle for every single spin in the ensemble,
		the formula can be reduced to
		\begin{equation}
			\hat\rho_\tau = \hat R_\phi(\beta) \hat \rho \hat R_\phi(-\beta)
		\end{equation}
		meaning a rotation of the magnetization corresponds to a rotation of the desity matrix.
		If we consider a $\SI{90}{\degree}$ pulse around the x axis on the previously described
		equilibrium for a spin-1/2 ensemble it follows:
		\begin{equation}
			\begin{split}
				\hat\rho_\tau = \hat R_x(\pi/2)\hat\rho\hat R_x(-\pi/2) &= \frac{1}{2} \hat 1 +
				\frac{1}{2} \mathbb{B}\hat R_x(\pi/2) \hat I_z \hat R_x(-\pi/2)\\
				&=
				\begin{pmatrix}
					\frac{1}{2} & -\frac{1}{4i}\mathbb{B}\\
					\frac{1}{4i}\mathbb{B}\ & \frac{1}{2}
				\end{pmatrix}
			\end{split}
		\end{equation}
		It can be equally shown that a $\SI{180}{\degree}$ pulse inverts the populations of the
		diagonal elements while the coherences are untouched.
		If the system is not in thermal equilibrium, the states will evolve and generally relax back
		into the original equilibrium state. If one neglects said relaxation at first, the evolution
		can be described by \todo{ref formula rotationg frame}
		\begin{equation}
			\begin{split}{ccc}
				\ket\psi_\upsilon &= \hat R_z(\Omega^0\upsilon)\ket\psi_\tau ~ \mathrm{and} \\
				\hat\rho_\upsilon &= \hat R_z(\Omega^0\upsilon)\hat\rho_\tau\hat
				R_z(-\Omega^0\upsilon)
			\end{split}
		\end{equation}
		which shows that only a time dependent phase $\exp{i\Omega^0 \upsilon}$ is added to the coherences and the populations
		stay constant. This confirms the macroscopic observation that the ensemble of spins behaves like a rotating vector of
		magnetization in the x-y-plane (w\textbackslash o relaxation).
		If relaxation is added to the scheme, we can differentiate between relaxation of the
		coherences ($T_2$ relaxation) and that of the states ($T_1$ relaxation).
	\section{NMR}
	Nuclear Magnetic Resonance (NMR) is a technique that emerged in 1946 with the discovery
	of the absortion properties of nuclei irradiated with electromagnetic waves resonant to their
	larmor frequencies.\ref{ResonanceAbsorption} The subsequent observation of signal from those
	previously excited nuclei, known as free induction decay (FID) paved the road to the now well
	established method which in the beginning was primarily used in chemistry for structural
	analysis of Molecules and chemical kinetics.
		\subsection{Flip angle}
			An external, on-resonant magnetic field will cause the magnetization of an ensemble of
			spins to flip by an angle $\alpha$ known as the flip angle (FA). An FA of $n\cdot 180
			\deg + 90 \deg$ will rotate the magnetization to the transverse plane perpendicular to the magnetic field
			resulting in an FID. Pulses of $n\cdot 180 \deg$ will keep the magnetization aligned
			with the magnetic field or invert it, therefore not resulting in a FID. All other FAs
			will produce a linear combination of the two cases.
		\subsection{Relaxation}
			After being deflected from its equilibrium, z-magnetization tends to relax towards its
			thermal equilibrium. That process is called T1 relaxation or spin-lattice relaxation and
			is an exponential decay. In addition, the transverse magnetization decays - usually much
			faster - because of different magnetic fields experienced by the individual spins
			leading to a dephasing of the signal.
			\begin{figure}{h}
				\todo{T1/T2 figure}
			\end{figure}
		\subsection{A simple NMR experiment}
		The most basic experiment in NMR is the exposure of a spin ensemble to a $\SI{90}{\degree}$
		RF pulse and subsequent readout. This will set the diagonal density matrix entries to zero
		and populate the socalled coherences, i.e. the off-diagonal entries.
		\subsection{Field Gradients}
		For different purposes it is beneficial not to have only the homogeneous $B_0$ field, but
		also field gradients. Generally, these are additional z-fields that change linearly with one
		of the spatial dimentions. Generation of these fields is usually achieved by thermal
		conductors of specifically tailored geometries.
		\subsection{2D-NMR}
	\section{MRI}
		\subsection{MRI in Medicine}
		\subsection{Frequency encoding}
		\subsection{Phase encoding}
	\section{Hyperpolarization}
		The main limitation of NMR and MRI is the low thermal polarization.
		For each energy state $E_i$ the Boltzmann distribution dictates
		\begin{equation}
			N_i = \exp{\frac{E_i}{k_B T}}
		\end{equation}
		For energy differences between two states that are small compared to
		the thermal Energy, the polarization can be expressed as follows:
		\begin{equation}
			P = {N_+-N_-}{N} = \tanh\left(\frac{\hbar \gamma B}{2 k T }\right)
		\end{equation}
		\subsection{DNP}
		\subsection{Hyperpolarization of Noble Gases}
		\subsection{Brute Force Hyperpolarization}
		\subsection{Parahydrogen induced Hyperpolarization}
