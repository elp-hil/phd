% !TEX root = ../thesis_main.tex
\chapter{Theory}\label{chap:theory}
	\section{NMR}
		Nuclear Magnetic Resonance (NMR) is a technique that emerged in 1946 with the discovery
		of the absortion properties of nuclei irradiated with electromagnetic waves resonant to their
		larmor frequencies.\cite{ResonanceAbsorption} The subsequent observation of signal from those
		previously excited nuclei, known as free induction decay (FID) paved the road to the now well
		established method which in the beginning was primarily used in chemistry for structural
		analysis of Molecules and chemical kinetics.
		\subsection{Larmor frequency}
			Inside an external magnetic field $B_0$, a magnetic dipole $\mu$ will precess with a frequency
			proportional to $B_0\cdot \mu$. This also applies to charged particles with a spin $S\neq0$ and thus
			a magnetic moment $\mu\neq0$. The precession is a result of the torque $\mu\times\vec B$
			excerted on the magnetic dipole moment.
		\subsection{Chemical shift}
			The larmor frequency of nuclei in NMR is largely governed by $B_0$ and $\mu$, but other factors do
			influence it on a more subtle scale. The most prominent effect is the chemical shift which
			originates in the shielding of $B_0$ by the electrons' magnetic dilpoles surrounding the nucleus or molecule.
			The shielding thus generally leads to a decrease of precession frequency, the chemical shift
			though is calculated as the relative frequency quota towards another known sample's
			reference frequency:
			\begin{equation}
				\delta = \frac{\nu_{sample} - \nu_{reference}}{\nu_{reference}}
			\end{equation}
		\subsection{J-coupling}
			In addition to the change in field by the electrons dipoles, the nucleis' magnetic
			dipoles inside one molecule will also interact. The interaction can be conveyed either
			directly or via the electrons' spins. The direct interaction is neglectable in liquids
			due to fastly rotating spins that average to zero. The indirect interaction via the
			electrons can also lead to a frequency shift resulting in multiplets usually on scales
			smaller than the chemical shifts, i.e. in the $\SI{}{\hertz}$ range.
		\subsection{Flip angle}
			An external, on-resonant magnetic field will cause the magnetization of an ensemble of
			spins to flip by an angle $\alpha$ known as the flip angle (FA). An FA of $\mathrm n\cdot 180
			\deg + 90 \deg$ will rotate the magnetization to the transverse plane perpendicular to the magnetic field
			resulting in an FID. Pulses of $\mathrm n\cdot 180 \deg$ will keep the magnetization aligned
			with the magnetic field or invert it, therefore not resulting in a FID. All other FAs
			will produce a linear combination of the two cases.
		\subsection{A simple NMR experiment}
		The most basic experiment in NMR is the exposure of a spin ensemble to a $\SI{90}{\degree}$
		RF pulse and subsequent readout. This will flip the magnetization by $\SI{90}{\degree}$
		which will subsequently precess around the z axisi with its Larmor frequency. Using a coil\ref{materialsMethods:coils}
		mounted around the sample, the alternating field caused by the rotating magnetization will induce a
		current driving the resonant circuit. That signal - or free induction decay - can be
		recorded as the voltage over the resonant circuit.
		\subsection{Relaxation}
		After being deflected from its thermal equilibrium, the magnetization tends to return to
		that equilibrium state. That process is called $T_1$ relaxation or spin-lattice relaxation
		and is an exponential decay. In addition, the transverse magnetization decays with a time
		constant called $T_2$ that is usually much smaller than $T_1$. $T_2$ relaxation is because
		of different magnetic fields experienced by the individual spins leading to a dephasing of
		the signal.
		\begin{figure}{h}
			\todo{T1/T2 figure}
		\end{figure}
		\subsection{Field Gradients}
		For different purposes it is beneficial to not only have the homogeneous $B_0$ field, but
		also field gradients. Generally, these are additional z-fields that change linearly with one
		of the spatial dimentions. Generation of these fields is usually achieved by thermal
		conductors of specifically tailored geometries.
		\subsection{2D-NMR}
		
	\section{Nuclei and Spins}
	The postulation of an electron spin in 1926 \ref{} and later in 1928 of a proton spin \ref{} opened
	up new areas of research.\todo{more history}
	Individual spins can be described by their angular momentum operators $\hat{I}$,
	$\hat{I}_y$ and $\hat{I}_z$ that return the eigenvalue of the state in its respective
	direction.
	\begin{equation}
		\hat I_z \ket{I, S} = S \ket{I,S}
	\end{equation}
	Considering a spin-1/2 particle in a magnetic field along the z-axis (e.g. the proton as a prominent particle in NMR), these
	eigenstates are
	\begin{equation}
	\ket\alpha = \begin{pmatrix}1\\0\end{pmatrix} \hspace{2 cm} \ket\beta =
	\begin{pmatrix}0\\1\end{pmatrix}
	\end{equation}
	with the eigenvalues of $\pm 1/2$.
	Generally, every spin-1/2 particle can be in any superposition of those two eigenstates:
	\begin{equation}
		\ket\psi = c_\alpha \ket\alpha + c_\beta \ket \beta = \begin{pmatrix} c_\alpha \\
		c_\beta\end{pmatrix}
	\end{equation}
	Th;lkjsdfljk:
	thtlth:wqa
	:ese superposition states evolve under external conditions until a eigenstate of the particle
	is reached.
	Generally, a sample does not consist of one, but of many nuclei and thus many spins need to be
	described to describe the system. To do so, usually the density matrix formalism is introduced.
	It onsiders an ensembe of spins that do not interact with each other. In this case, the
	expectation value of an operator for a single spin $\braket{\hat Q}$ is given by
	\begin{equation}
	\bra\psi\hat Q\ket\psi = \left( c_\alpha^*, c_\beta^*\right)
	\begin{pmatrix}
		Q_{\alpha\alpha} Q_{\alpha\beta}\\
		Q_{\beta\alpha} Q_{\beta\beta}
	\end{pmatrix}
	\begin{pmatrix}
		c_\alpha\\
		c_\beta
	\end{pmatrix}
	\end{equation}
	It can be shown that this is equal to the trace of the density operator
	$\ket\psi\bra\psi$multiplied with said operator
	\begin{equation}
		\braket{\hat Q} = \Tr \left\{\ket\psi\bra\psi\hat Q\right\}
	\end{equation}
	For multiple spins, it follows that the observable becomes the sum of their individual
	observables:
	\begin{equation}
		\braket{\hat Q_{all}}= \bra{\psi_1}\hat Q\ket{\psi_1} + \bra{\psi_2}\hat Q\ket{\psi_2} + \hdots =
		\Tr{\left\{\left(\ket{\psi_1}\bra{\psi_1} + \bra{\psi_2}\ket{\psi_2}+ \hdots \right) \hat Q\right\}}
	\end{equation}
	This means that the system of spins can be described by the sum of the density operators called
	the density matrix
	\begin{equation}
		\hat\rho = \overline{\ket\psi\bra\psi}.
	\end{equation}
	For a spin-1/2 ensemble, the density matrix in thermal equilibrium is
	\begin{equation}
		\hat \rho = \begin{pmatrix} \frac{1}{2}+\frac{1}{4}\mathbb{B}& 0\\ 0&
		\frac{1}{2}-\frac{1}{4}\mathbb{B}\end{pmatrix} = \frac {1}{2} \hat1 + \frac{1}{2} \mathbb{B}
		\hat I_z
	\end{equation}
	with $\mathbb{B} = \frac{\hbar\gamma B_0}{k_b T}$. That means that in thermal equilibrium, the
	non-diagonal elements are zero, i.e. the socalled coherences (off-diagonal elements) are all
	equally populated while there is a slight overpopulation of one of the two states. Through deflection from that
	equilibrium, coherences are populated as described in the following section.
		\subsection{Radiofrequency pulses}
		Using the density matrix formalism, radiofrequency pulses described by rotational operators can be
		applied to the whole ensemble. A pulse $\hat R_\phi(\beta)$ of phase $\phi$ (corresponding
		to the axis around which magnetization is rotated) and angle $\beta$ where
		$\beta=\omega_{nut} \cdot \tau_p$ exerting on a state $\ket\psi$ is described by 
		\begin{equation}
			\ket{\psi_\tau}= \hat R_\phi(\beta)\ket\psi
		\end{equation}
		Calculating the density matrix now leads to
		\begin{equation}
			\hat {\rho_\tau} = \overline{\ket{\psi_\tau}\bra{\psi_\tau}} = \overline{\hat
				R_{\phi}(\beta)\ket\psi\bra\psi \hat R_\phi(-\beta)}
		\end{equation}
		where the overbar describes the averaging over all spins in the ensemble.
		Finally, as we are considering the same flip angle for every single spin in the ensemble,
		the formula can be reduced to
		\begin{equation}
			\hat\rho_\tau = \hat R_\phi(\beta) \hat \rho \hat R_\phi(-\beta)
		\end{equation}
		meaning a rotation of the magnetization corresponds to a rotation of the desity matrix.
		If we consider a $\SI{90}{\degree}$ pulse around the x axis on the previously described
		equilibrium for a spin-1/2 ensemble it follows:
		\begin{equation}
			\begin{split}
				\hat\rho_\tau = \hat R_x(\pi/2)\hat\rho\hat R_x(-\pi/2) &= \frac{1}{2} \hat 1 +
				\frac{1}{2} \mathbb{B}\hat R_x(\pi/2) \hat I_z \hat R_x(-\pi/2)\\
				&=
				\begin{pmatrix}
					\frac{1}{2} & -\frac{1}{4i}\mathbb{B}\\
					\frac{1}{4i}\mathbb{B}\ & \frac{1}{2}
				\end{pmatrix}
			\end{split}
		\end{equation}
		It can be equally shown that a $\SI{180}{\degree}$ pulse inverts the populations of the
		diagonal elements while the coherences are untouched.
		If the system is not in thermal equilibrium, the states will evolve and generally relax back
		into the original equilibrium state. If one neglects said relaxation at first, the evolution
		can be described by \todo{ref formula rotationg frame}
		\begin{equation}
			\begin{split}
				\ket\psi_\upsilon &= \hat R_z(\Omega^0\upsilon)\ket\psi_\tau ~ \mathrm{and} \\
				\hat\rho_\upsilon &= \hat R_z(\Omega^0\upsilon)\hat\rho_\tau\hat
				R_z(-\Omega^0\upsilon)
			\end{split}
		\end{equation}
		which shows that only a time dependent phase $\exp{i\Omega^0 \upsilon}$ is added to the coherences and the populations
		stay constant. This confirms the macroscopic observation that the ensemble of spins behaves like a rotating vector of
		magnetization in the x-y-plane (w\textbackslash o relaxation).
		If relaxation is added to the scheme, we can differentiate between relaxation of the
		coherences ($T_2$ relaxation) and that of the states ($T_1$ relaxation). The former affects
		the non diagonal elements only which will deacy to zero while the latter brings the populations of the states back to
		thermal equilibrium. That can be expressed by
		\begin{equation}
			\rho_{ij, \upsilon} = \rho_{ij, \tau} \exp{(\pm
				i\Omega^0-\lambda)\upsilon},~\mathrm{i,j~=~
			1,2(+)~or~2,1(-)}
		\end{equation}
		for the non diagonal elements and
		\begin{equation}
			\rho_{ii,\upsilon} = (\rho_{ii,\tau} - \rho_{ii}^{eq})\exp(-\tau/T_1)+\rho_{ii}^{eq}
		\end{equation}

	\section{MRI}
		A huge addition to the world of NMR was the invention of spatially resolved sample maps,
		i.e. imaging. It was made possible through the clever use of gradients and largely resembles
		2D-NMR sequences.
		\subsection{MRI in Medicine}
			As a non-invasive, non-ionizing imaging method, MRI has become important in many parts
			of the medical field for example neurology, oncology, cardiology or urology. While the
			low polarization of nuclei make its sesitivity inferior to other imaging methods such as
			computed tomography, the nature of the signal generation allow for contrasts that differ
			vastly from other established mehtods. This uniqueness allows for superior imaging power
			in many cases despite the smaller sensitvity.
		\subsection{Slice selection}
			The first step in most imaging methods is to excite a slice of the sample to perform the
			first spatial selection. To do so, a gradient - mostly in z direction - is aplied to the
			sample. The following pulses which usually have a limited frequency bandwidth will
			affect only the nuclei within a slice of the thickness defined by the gradient strength
			usually given in $\SI{}{\milli\tesla\per\meter}$ and the pulse's bandwith. It has to be
			kept in mind that this does not necessarily guarantee that nuclei outside the prepared
			slice do not contribute to the signal. Movement of parts of the sample such as flow or
			diffusion can lead to signal artifacts. Furthermore, the following pulses in the
			sequence have to be carefully designed to not generate FIDs or echos of their own that
			are not related to the originally selected slice.
		\subsection{Frequency encoding}
			During readout, a field gradient can be turned on to encode the second spatial
			dimension. Considering a positive x-gradient, the nuclei at higher x values will then
			generate a higher frequency than the ones at lower x values and thus show up further
			right in a 1D-FT.
		\subsection{Phase encoding}
			Generating an encoding for the third spatial dimension is not quite as straightforward.
			To do so, a third gradient in a third dimension perpendicular to the two others is necessary.
			Due to redundancy issues, that gradient cannot simply be turned on during readout though
			but is used to change the phase of the signal the nuclei generate depending on the
			position in the sample. That means that the gradient strength needs to be varied to
			satisfy the sampling conditions for the frequencies generated, i.e. the freuquency
			encoding scheme is run multiple times with a different phase encoding gradient strengths running
			before it. That way, if the fourier transformed data for each phase encoding step is
			sorted by gradient strength and fourier transformed in the second dimension, the
			frequency generated by the more or less quickly changing phase will define the position
			in y direction.
	\section{Hyperpolarization}
		The main limitation of NMR and MRI is the low thermal polarization.
		For each energy state $E_i$ the Boltzmann distribution dictates
		\begin{equation}
			N_i = \exp{\frac{E_i}{k_B T}}
		\end{equation}
		For energy differences between two states that are small compared to
		the thermal Energy, the polarization can be expressed as follows:
		\begin{equation}
			P = {N_+-N_-}{N} = \tanh\left(\frac{\hbar \gamma B}{2 k T }\right)
		\end{equation}
		\subsection{Dynamic nuclear polarization}
			As the gyromagnetic ratio of electrons is much higher than that of any nucleus (factor
			of $10^{3}$ compared to protons), dynamic nuclear polarization (DNP) uses electrons at
			low temperatures and high fields to generate large initial polarization. The
			polarization is then transferred to specific nuclei by microwave irradiation that
			induces transitions in the spin system of nucleus and electron that is generated by the
			addition of free radicals. The low temperatures lead to a freezing of the sample and
			microwaves are applied in that frozen state. For administration, the frozen sample then
			needs to be quickly melted and transferred, usually in controlled magnetic environments
			to prevent fast relaxation. The samples can show high polarizations close to unity but
			are usually limited by decay during the relatively long delivery times. Efforts building
			faster transport mechanisms have greatly reduced these times but they stay in the range
			of tens of seconds.
		\subsection{Hyperpolarization of Noble Gases}
		\subsection{Brute Force Hyperpolarization}
		\subsection{Parahydrogen induced Hyperpolarization}
