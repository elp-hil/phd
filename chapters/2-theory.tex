\chapter{Theory}\label{chap:theory}
	\section{Nuclei and Spin}
	The postulation of an electron spin in 1926 \ref{} and later in 1928 of a proton spin \ref{} opened
	up new areas of research.\todo{more history}
	Individual spins can be described by their angular momentum operators $\hat{I}$,
	$\hat{I}_y$ and $\hat{I}_z$ that return the eigenvalue of the state in its respective
	direction.
	\begin{equation}
		\hat I_z \ket{I, S} = S \ket{I,S}
	\end{equation}
	Considering a spin-1/2 particle in a magnetic field along the z-axis (e.g. the proton as a prominent particle in NMR), these
	eigenstates are 
	\begin{equation}
	\ket\alpha = \begin{pmatrix}1\\0\end{pmatrix} \hspace{2 cm} \ket\beta =
	\begin{pmatrix}0\\1\end{pmatrix}
	\end{equation}
	with the eigenvalues of $\pm 1/2$.
	Generally, every spin-1/2 particle can be in any superposition of those two eigenstates:
	\begin{equation}
		\ket\psi = c_\alpha \ket\alpha + c_\beta \ket \beta = \begin{pmatrix} c_\alpha \\
		c_\beta\end{pmatrix}
	\end{equation}
	These superposition states evolve under external conditions until a eigenstate of the particle
	is reached.
	Generally, a sample does not consist of one, but of many nuclei and thus many spins need to be
	described to describe the system. To do so, usually the density matrix formalism is introduced.
	It onsiders an ensembe of spins that do not interact with each other. In this case, the
	expectation value of an operator for a single spin $\braket{\hat Q}$ is given by
	\begin{equation}
	\bra\psi\hat Q\ket\psi = \left( c_\alpha^*, c_\beta^*\right)
	\begin{pmatrix}
		Q_{\alpha\alpha} Q_{\alpha\beta}\\
		Q_{\beta\alpha} Q_{\beta\beta}
	\end{pmatrix}
	\begin{pmatrix}
		c_\alpha\\
		c_\beta
	\end{pmatrix}
	\end{equation}
	It can be shown that this is equal to the trace of the density operator 
	$\ket\psi\bra\psi$multiplied with said operator
	\begin{equation}
		\braket{\hat Q} = \Tr \left\{\ket\psi\bra\psi\hat Q\right\}
	\end{equation}
	For multiple spins, it follows that the observable becomes the sum of their individual
	observables:
	\begin{equation}
		\braket{\hat Q_{all}}= \bra{\psi_1}\hat Q\ket{\psi_1} + \bra{\psi_2}\hat Q\ket{\psi_2} + \hdots =
		\Tr{\left\{\left(\ket{\psi_1}\bra{\psi_1} + \bra{\psi_2}\ket{\psi_2}+ \hdots \right) \hat Q\right\}}
	\end{equation}
	This means that the system of spins can be described by the sum of the density operators called
	the density matrix
	\begin{equation}
		\hat\rho = \overline{\ket\psi\bra\psi}.
	\end{equation}
	In thermal equilibrium, the non-diagonal elements are zero. Through deflection from that
	equilibrium, non diagonal elements are populated as described in the following section.
	\section{NMR}
	Nuclear Magnetic Resonance (NMR) is a technique that emerged in 1946 with the discovery
	of the absortion properties of nuclei irradiated with electromagnetic waves resonant to their
	larmor frequencies.\ref{ResonanceAbsorption} The subsequent observation of signal from those
	previously excited nuclei, known as free induction decay (FID) paved the road to the now well
	established method which in the beginning was primarily used in chemistry for structural
	analysis of Molecules and chemical kinetics.
		\subsection{Flip angle}
			An external, on-resonant magnetic field will cause the magnetization of an ensemble of
			spins to flip by an angle $\alpha$ known as the flip angle (FA). An FA of $n\cdot 180
			\deg + 90 \deg$ will rotate the magnetization to the transverse plane perpendicular to the magnetic field
			resulting in an FID. Pulses of $n\cdot 180 \deg$ will keep the magnetization aligned
			with the magnetic field or invert it, therefore not resulting in a FID. All other FAs
			will produce a linear combination of the two cases.
		\subsection{Relaxation}
			After being deflected from its equilibrium, z-magnetization tends to relax towards its
			thermal equilibrium. That process is called T1 relaxation or spin-lattice relaxation and
			is an exponential decay. In addition, the transverse magnetization decays - usually much
			faster - because of different magnetic fields experienced by the individual spins
			leading to a dephasing of the signal.
			\begin{figure}{h}
				\todo{T1/T2 figure}
			\end{figure}
		\subsection{A simple NMR experiment}
			The most basic experiment in NMR is the reaction of a spin ensemble to an 
		\subsection{Field Gradients}
		\subsection{2D-NMR}
	\section{MRI}
		\subsection{MRI in Medicine}
		\subsection{Frequency encoding}
		\subsection{Phase encoding}
	\section{Hyperpolarization}
		The main limitation of NMR and MRI is the low thermal polarization.
		For each energy state $E_i$ the Boltzmann distribution dictates
		\begin{equation}
			N_i = \exp{\frac{E_i}{k_B T}}
		\end{equation}
		For energy differences between two states that are small compared to
		the thermal Energy, the polarization can be expressed as follows:
		\begin{equation}
			P = {N_+-N_-}{N} = \tanh\left(\frac{\hbar \gamma B}{2 k T }\right)
		\end{equation}
		\subsection{DNP}
		\subsection{Hyperpolarization of Noble Gases}
		\subsection{Brute Force Hyperpolarization}
		\subsection{Parahydrogen induced Hyperpolarization}
