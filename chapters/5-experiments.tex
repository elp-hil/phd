\chapter{Results}\label{chap:results}
	As a large part of this work consisted of hardware design and manufacturing, I will split the
	results section into sole hardware results and measurements with said hardware.
\section{Hardware}
	\subsection{Low field NMR}
		\subsubsection{Receive coils}
			Receive circuits provided q-factors of \todo{measure} resulting in resonances about
			$\SI{1}{\kilo\hertz}$ wide. Note that the connection to the NI DAQ introduces additional
			capacities and inductances that cannot be neglected especially in the higher frequency
			range above $\SI{100}{\kilo\hertz}$ where the coil's intrinsic capacities are
			comparatively small.
		\subsubsection{$\mathbf{B_0}$ coils}
			The manufactured $\mathrm{B_0}$ coils show fields (i.e. frequencies) as well as  linewidths in the order of magnitude the
			simulations predicted. Through manufacturing errors, and changes to the coil in long
			term use cases, the linewidths deteriorated. It is important to note that, at currents
			above $\SI{1}{\ampere}$, the $\mathrm{B_0}$ coil heated noiticabely. while the heating
			itself is unproblematic, the resulting exansion of the materials leads to an overall
			longer coil and thus lower fields \ref(fieldSolenoid). To avoid field shifts during
			measurements, the setup should therefore be in thermal equilibrium. This is especially
			relevant when switching fields during measurements using the programmable power supply.
		\subsection{Shims and programmable power supply}
			The three linear shims are settable via the programmable power supply and show
			their expected effect. The rotational position of the receive coil is relevant to the
			shims indicating they're working as intended. Using the added shim tool, line widths
			were reduced from $\approx \SI{250}{\hertz}$ to $\approx \SI{30}{\hertz}$ in the case of large samples of $\approx
			$\SI{40}{\milli\litre}. If the initial field was generated by a less homogeneous, more
			asymmetric coil in which no signal was visible without shims, signal was discovered and linewidths went down to
			$\approx \SI{50}{\hertz}$.
	\subsection{Sabre shuttling system}
		The system designed to transfer a sample between fields works as intended. Fluid losses are
		small and acceptable with about \todo{how much} lost within \todo{nn} shuttling cycles. The
		bubbling system works well and provides pH2 to the solution in large amounts. The pressure
		stability of the vessels has been tested to withstands up to $\SI{50}{\bar}$. Chemical
		resistance is good though resistance to pure pyridine is not given. While there was never
		any problem with the $\SI{}{\milli\Molar}$ pyridine concentrations used in the experiments, a
		neat pyridine batch showed to dissolve the PSU casing.
	\subsection{Fluxgate readout electronics}
		The readout electronics were designed to feature a wide range of amplifications for all
		three spatial dimensions. A 24 V DC power supply was fitted with a DC-DC-converter to
		provide the $\pm\SI{15}{\volt}$ to power the Fluxgate. Additionally the PCB board was fitted
		with the electronic parts (see \ref{sec:methods}). For testing purposes, a simple program
		using serial in and output to toggle the analog switches was written. All switches were
		successfully tested to work, though three had to be replaced at some point for malfunction.
		Apparently this was due to damage during assembly as the board is now working as expected.
\section{Measurements}
	\subsection{Low field NMR}
	\subsection{Sabre in water}
	\subsection{Sabre in cell solution and blood}
	\subsection{15N Sabre}
	\subsection{Nanotesla field measurements}
	\subsection{High field Sabre}

%\input{figures/experiments/figExample}
