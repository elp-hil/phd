% !TEX root = ../thesis_main.tex
\chapter{Simulations}\label{chapter:simulations}
\section{Biot Savart field simulations}
\subsection{$B_0$ fields in low field NMR}\label{simulations:B0}
        For $B_0$ field generation, mostly solenoid coils with compensation windings on both ends were used. To find optimal lengths and numbers of layers, Matlab simulations of static fields using Biot Savart's law 
        \begin{equation}
            B(\vec r) = \frac{\mu_0}{4\pi} \int_V I\mathrm{d} \vec l \times \frac{\vec r - \vec r^{'}}{\left|\vec r - \vec r^{'}\right|^3}
        \end{equation}
        To calculate the fields, the coils were split into 30 parts per turn and each part was considered a straight, short enough path for numerical integration. Two loops stepped through the windings and winding parts while the calculation for each part was performed in matrix notation for the cube of interest. This sped up the calcualtions compared to looping in Matlab.Adittionally, a parallel pool was added to reduce calculation time for the inner loop over the winding parts.
        were implemented.  Both solenoidal "corcsrew" structures and closed loops were simulated. The measure considered was absolute field distribution inside a volume of $\SI{3}{\centi\meter}$ by  $\SI{3}{\centi\meter}$ by $\SI{3}{\centi\meter}$.  Fields were calculated in a grid of \todo{30 cubed} points. Absolute field difference inside the volume was considered first to find the approximate range in which homogeniety is high.  It was then refined by histogram evaluation which yields results similar to the lorentian distribution a fourier transformed NMR signal shows. To estimate linewidths generated by field inhomogenyeties, the conservative measure of the central 80 \% of field points was used.  Using this representation and varying different parameters such as diameter, distances and relative currents in different loop structures, optimal geometries for different coil designs were calculated.
        \subsection{Solenoid Coil}
            The field of an infinitely long solenoid would be almost perfectly homogeneous along its symmetry axis. Due to the limited length of the actual coil, even on the symmetry axis, the field diverges toward the ends of the coil.
        \subsubsection{Additional Compensation Windings}
            Additional compensation windings further smoothed the field by adding a differently shaped field component that can be used to partly compensate the solenoids inhomogenieties. Figure \todo{simulate} shows the fields of solenoid and and compensation windings in the x-z-plane and the resulting superposition.
            To evaluate the field in the 3D sample volume, histograms of the field distribution were generated for different compensation wind lenghts. As visible in figure nn, the influence of a single wind are quite substantial indicating that the manufacturing process must be well controlled. Additionally, it can be seen that

        \subsection{Dual Helmholtz Array}\label{simulations:DualHelmholtzArray}
        As a more advanced setup that is less prone to manufacturing errors due to the relatively larger distances to the sample volume, a dual helmholtz array design was considered. A single Helmholtz array provides less Homogeniety than a solenoid \ref{}, but the additional pair of coils generates a differently shaped field similar to the compensation windings for the solenoid. By variation of the distances of both pairs relative to the center and their respective field strengths, i.e. the currents of the coil pairs, optimal parameters were extracted from the simulations. As the maximum and minimum radii were given by the space between mu metal shield and the shimming tube between which the B0 coil should be positioned, the variation in radius was not considered. Instead, coils were predetermined to be of inner radius \SI{65}{\mm} with twelve layers width and eight layers height.
        Aditionnaly, absolute fields were considered to ensure that fields for 1H Sabre could be generated with the coils.
    \section{LTSpice electronics simulations}
        For designing coils and resonant circuits, simulations of the circuits could make sense to double check circuits for errors.In cases of single resonant coils, this was not necessary, but for the design of dual resonant coils already, this could make sense. Parts can be described very well in this software, even close to reality as blind inductances and capacitances can be added to the parts as well as ohmsch resistances for parts that ideally would be of solely complex impedance.
        A design is first drawn, values are then added on a part by part basis and the program can then run differen analyses of the circuit. This includes AC analyses as well as DC calculations. Different stepsizes can be used aand different frequencies can be sampled. The flexibility is high as the input - while featuring a GUI - is actually pseudo-command line based. Circuits such as the first arduino shield for the fluxgate were simulated to ensure proper readout of voltages. For the dual resonant, single channel circuits, different designs have been tested to find optimally performing compositions.
    \section{Autodesk Inventor simulations}
        The simulation toolbox of Autodesk Inventor was used to check pressure resistances of parts virtually created before manufacture. The program is easy to use - Definitions of the materials used and the volume encased were necessary, the former is given in table \ref{} For PSU and PVC, the two materials mainly used and was taken from \ref{}. The parameters lower bounds were entered for strengths to ensure that in worst case scenarios, the parts would still hold. With the parameters entered and a closed surface defined (simple point and click, if definition is difficult, use the sidebar to select parts) Autodesk used a finite elements method to calculate the strain on the components. Display of the results is in color, but actual deformation is displayed strongly exaggerated for better visualization of the deformation that is usually small compared to the dimensions of the parts.
    \section{Simulations using the groups' spin dynamics\todo{name?} framework}
        To estimate the field and contact times necessary for the Sabre experiments with nicotinamide, calculations previously done in this group for pyridine \ref{stephan} were repeated for nicotinamide. Thecoupling of the sample molecule to the pH2 via the catalyst was kept \ref{paper duckett} while the chemical shifts inside the molecule and thhus the coupling between the internal spins were changed. Parameters are listed in \ref{table:simulations:spinFrameworkParameters}
        \begin{table}
        \centering
            \begin{tabular}{ccc}
                Parameter & \\
                Value & 
            \end{tabular}
            \caption[Spin framework parameters]{Parameters used in the spin framework simulations. Couplings between pH2 and catalyst from \ref{duckett paper}. }
        \end{table}
