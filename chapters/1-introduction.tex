\chapter{Introduction}\label{chap:introduction}
\section{setup.tex}\label{sec:setup}
Edit setup.tex according to your needs. The file contains two sections, one for package includes, and one for defining commands. At the end of the includes and commands there is a section that can safely be removed if you don't need algorithms or tikz. Also don't forget to adapt the pdf hypersetup!!\\
setup.tex defines:
\begin{itemize}
    \item some new commands for remembering to do stuff:
    \begin{itemize}
        \item \verb|\todo{Do this!}|: \todo{Do this!}
        \item \verb|\extend{Write more when new results are out!}|:\\ \extend{Write more when new results are out!}
        \item \verb|\draft{Hacky text!}|: \draft{Hacky text!}
    \end{itemize}

    \item some commands for referencing, `in \verb|\chapref{chap:introduction}|' produces 'in \chapref{chap:introduction}'
    \begin{itemize}
        \item \verb|\chapref{}|
        \item \verb|\secref{sec:XY}|
        \item \verb|\eqref{}|
        \item \verb|\figref{}|
        \item \verb|\tabref{}|
    \end{itemize}

    \item the colors of the Uni's corporate design, accessible with\\ \verb|{\color{UniX} Colored Text}|
    \begin{itemize}
        \item {\color{UniBlue}UniBlue}
        \item {\color{UniRed}UniRed}
        \item {\color{UniGrey}UniGrey}
    \end{itemize}

    \item a command for naming matrices \verb|\mat{G}|, $\mat{G}$, and naming vectors \verb|\vec{a}|, $\vec{a}$. This overwrites the default behavior of having an arrow over vectors, sticking to the naming conventions  normal font for scalars, bold-lowercase for vectors, and bold-uppercase for matrices.

    \item named equations:
        \begin{verbatim}
\begin{align}
    d(a,b) &= d(b,a)\\ \eqname{symmetry}
\end{align}
        \end{verbatim}
        \begin{align}
            d(a,b) &= d(b,a)\\ \eqname{symmetry}
        \end{align}
\end{itemize}
