% !TEX root = ../thesis_main.tex
\chapter{Introduction}
\label{chap:introduction}
    \section{Hyperpolarization}
    This work is all about hyperpolarization (HP), which can be described as the magnetization of spins exceeding their thermal equilibrium magnetization levels. The range of methods for reaching this goal is wide \cite{johannesson_dynamic_2009,hirsch_brute-force_2015,duhamel_xenon-129_2001,fain_imaging_2010, bowers_parahydrogen_1987-2, adams_reversible_2009-2}, this work is focused on chemically induced hyperpolarization using parahydrogen \cite{green_theory_2012-1}.
    The HP of nuclear spins promises to overcome MRI's greatest impediment, its low polarization and thus low sensitivity. In fact, $^{13}$C HP has been used successfully in metabolic MRI where it delivered significant insights into cancer metabolism in vivo \cite{golman_cardiac_2008}. Mostly, hyperpolarized small molecules in solution are produced externally in a polarizer \cite{ardenkjaer-larsen_present_2016}, followed by transfer, administration and imaging in a conventional MR system at magnetic field strengths of \SI{1.5}{\tesla} or more. From the moment of its production, HP decays with a characteristic time constant T$_1$ that depends on the magnetic field but rarely exceeds one minute. However, if the sample is exposed to a magnetic field below a critical value, then rapid relaxation will occur and the polarization is lost. Furthermore, during imaging, the transverse magnetization decays according to T$_2$ which is estimated to be of the order of \SI{100}{\second} to  \SI{0.1}{\second}. Thus, over time, polarization is lost to the in vivo measurement by readout, dilution, excretion and relaxation. Despite these challenges, impressive results have been obtained with respect to medical diagnostics in both mice and man through the injection of a single bolus of hyperpolarized agent. Whilst pyruvate is receiving great attention, no other agents have yet been injected into man as the regulatory approval process is difficult to cross. Progress to bypass relaxation is however prospering with research into long-lived magnetization of single nuclei like Lithium-64 \cite{van_heeswijk_hyperpolarized_2009} or Silicon-295 \cite{kwiatkowski_nanometer_2017} and long-lived quantum states of multi-atomic systems \cite{pileio_storage_2010, noauthor_y._nodate} being particularly noteworthy. Recently, a method to continuously hyperpolarize  nuclear spins by means of parahydrogen (pH2) and reversible exchange, Sabre, has emerged \cite{adams_reversible_2009-2, hovener_continuous_2014-1}. It offers the potential of repolarizing spins continuously in the presence of pH2. Its biological applicability, though, is limited because of the use of toxic methanol as the solvent as well as target molecules with limited biocompatibility. In this work it is shown that more biologically relevant agents can be hyperpolarized continuously in aqueous solution \cite{truong_irreversible_2014-1}. However, as we moved through experiments, it became clear that re-hyperpolarization in vivo would be difficult if not impossible to achieve. Therefore, the next step was to instead generate high batch magnetization (i.e. polarization above \SI{1}{\percent} at concentrations in the high \si{\milli\molar} range) instead of a lower, but renewable magnetization. This goal was aimed towards the same parahydrogen based SABRE method, and, as an additional step, the focus was shifted from $^1$H HP to X-nuclei, particularly $^{15}$N \cite{truong_15n_2015-1}. The advantage here is that relaxation times can be longer and there is no background signal from the usually widely available water in biomedical application. This aim lead to the creation of new hardware designs to optimize polarization parameters and thus increase magnetization to the highest achievable levels.
    \section{Hardware}
    The hardware used in HP experiments is often costly and complex to use. Dynamic nuclear polarization methods, for example, need high fields, low temperatures and careful handling of the substrate after polarization to generate high polarizations \cite{ardenkjaer-larsen_present_2016, milani_magnetic_2015}. Additionally, polarization times are long (with a few exceptions) often making long lead times necessary. The hardware in this work was designed to be easily and cheaply manufactured with ease of use and reproducibility in mind. Many aspects of hardware design are covered: 3D-CAD design of the components, fluid path and flow design for both gases and liquids, design of electronics for control of the setup and readout of sensor data as well as construction of components of the NMR and MRI setup such as coils for static magnetic field generation, radiofrequency-pulses and signal reception. Manufacturing methods mostly used after manual constructions were milling, lathing and 3D-printing. Using the equipment built in this work, high SABRE polarized samples were produced and measured in commercially available small animal MRI. If biologically relevant tracers are found, the setup can be used for generation of highly polarized batches of sample that in-vivo will portray metabolic processes.
