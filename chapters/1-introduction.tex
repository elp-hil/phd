% !TEX root = ../thesis_main.tex
\chapter{Introduction}\label{chap:introduction}
\section{Hyperpolarization}
This work is all about hyperpolarization (HP), which can be described as the magnetization of spins to levels exceeding their thermal equilibrium polarization. The range of moethods for reaching this goal is wide, this work focuses on chemically induced hyperpolarization.
The HP of nuclear spins promises to overcome MRI's greatest impediment, its low polarization and thus low sensitivity. In fact, $^{13}C$ HP has been used successfully in metabolic MRI where it delivered significant insights into cancer metabolism in vivo. To date, however, all hyperpolarized small molecules in solution are produced externally in a polarizer, followed by transfer, administration and imaging in a conventional MR system of \SI{1.5}{\tesla} or more. From the moment of its production, HP decays with a characteristic time constant $T_1$ that depends on the magnetic field but rarely exceeds one minute. However, if the sample is exposed to a magnetic field below a critical value, then rapid relaxation will occur and the polarization lost. Furthermore, during imaging, the transverse magnetization decays according to $T_2$ which is estimated to be of the order of \SI{100}{\second} to  \SI{0.1}{\second}. Thus, over time, polarization is lost to the in vivo measurement by readout, dilution, excretion and relaxation. Despite these challenges, impressive results have been obtained with respect to medical diagnostics in both mice and man through the injection of a single bolus of hyperpolarized agent. Whilst pyruvate is receiving great attention, no other agents have yet been injected into man as the regulatory approval process is difficult to cross. Progress to bypass relaxation is however prospering with research into long-lived magnetization of single nuclei like Lithium-64 or Silicone-295 and long-lived quantum states of multi-atomic systems being particularly noteworthy. Recently, a method to continuously hyperpolarize  nuclear spins by means of parahydrogen (pH2) and reversible exchange, Sabre, has emerged. It offers the potential of repolarizing spins continuously in the presence of pH2. Its biological applicability, though, was limited because of the use of toxic methanol as the solvent as well as target molecules with limited biocompatibility. In this work it is shown that more biologically relevant agents can be hyperpolarized in acqueous solution. Moreover, as we moved through experiments, it became clear that re-hyperpolarization in vivo would be difficult if not impossible to achieve. Therefore, the next step was to rather generate high batch magnetization instead of a lower, but renewable magnetization. This was aimed for using the same parahydrogen based Sabre method, and, as an additional step, the focus was shifted from $^1H$ HP to x-nuclei, particularly $^{15}N$. This lead to the creation of new hardware designs to optimize polarization parameters and thus increase magnetization to the highest possible levels.
\section{Hardware}
The hardware used in HP experiments is often costly and complex to use. Dynamic nuclear polarization methods, for example, need high fields, low temperatures and very careful handling of the substrate after polarization to generate high polarizations. Additionally, polarization times are long (with a few expceptions) often making long lead times necessary. The hardware in this work was designed to be easily and cheaply manufactured with ease of use and reproducibility in mind. Many aspects of hardware design are covered: 3D-CAD design of the components, fluid path and flow design for both gases and fluids, design of electronics for control and readout of setup and sensors as well as construction of components of the NMR and MRI setup such as coils for static magnetic field generation, rf-pulses and signal reception.
